\documentclass[11pt,a4paper]{article}
%\usepackage[latin1]{inputenc}   % Descomentar para compilar no windows
\usepackage[utf8x]{inputenc}     % Comentar para compilar no windows
\usepackage{graphicx}
\usepackage{svg}
\usepackage{color}
\usepackage{xspace}
\usepackage[brazil]{babel}
%\usepackage[latin1]{inputenc}
\usepackage{amsfonts, amssymb,amsmath}
\usepackage{array,booktabs}
\usepackage{proof}
\usepackage{fancyhdr}
\usepackage{paralist}
\usepackage{enumitem}
\usepackage{multicol}

%\usepackage[pdftex=true,pagebackref=true,bookmarks=true,bookmarksnumbered=true,pdffitwindow=true]{hyperref}
\usepackage{hyperref}
\usepackage{marginnote}
%\usepackage[top=Bcm, bottom=Hcm, outer=Ccm, inner=Acm, heightrounded, marginparwidth=Ecm, marginparsep=Dcm]{geometry}
\usepackage{comment}
\usepackage{tkz-euclide}

\newcounter{counterQuestoes}

\renewcommand{\b}{\textbf}
\newcommand{\noi}{\noindent}
\newcommand{\Z}{\mathbb{Z}}
\newcommand{\N}{\mathbb{N}}
\newcommand{\cis}[2]{{#1}\left(\cos {#2} + i \operatorname{sen} {#2} \right)}
\newcommand{\diff}[2]{\dfrac{\partial #1}{\partial #2}}

\setlength{\textwidth}{170mm}%
\setlength{\textheight}{259mm}%
\setlength{\topmargin}{-20mm}%
%\setlength{\bottommargin}{}%
\setlength{\oddsidemargin}{-10mm}
\setlength{\evensidemargin}{-30mm}%

\begin{document}
\frenchspacing
\begin{center}
    \begin{minipage}{4cm}
		\begin{center}
			\includegraphics[width=4cm, height=2.0cm]{logoifb.jpeg}
		\end{center}
	\end{minipage}
	\begin{minipage}{11.4cm}
		\begin{center}
				{\small \textsc{Instituto Federal de Brasília}			\\
						  \textsc{Campus Samambaia}\\
						  \textsc{3º ano - Ensino Médio Integrado} \\
                         \textbf{Disciplina:} Matemática \hspace{.65cm}\textbf{Ano:} 2019\\
                          \textsc{\textbf{Professor:} Bruno Xavier}\\
                }
		\end{center}
	\end{minipage}
	%\begin{minipage}{1.6cm}
	%	\begin{center}
	%		\includegraphics[width=2cm, height=1.1cm]{logoimd.png}
	%	\end{center}
	%\end{minipage}
\end{center}


{\sf
  \begin{center}
    \Large \textbf{--- Teste Objetivo - 1º bimestre ---}%
  \end{center}
}\bigskip

\setlength{\marginparwidth}{5cm}
\small \noindent \textbf{Nome:}\hspace{0.3cm}\hrulefill \hrulefill
\hrulefill \hspace{0.1cm}
\textbf{Turma:}\hspace{0.1cm}\rule{1cm}{.1mm} \textbf{Nota:}\hspace{0.1cm}\rule{1cm}{.1mm}\\*[0.1cm]
%\small \noindent %\textbf{Nome:}\hspace{0.3cm}\hrulefill \hrulefill \hrulefill \hspace{0.1cm}


\thispagestyle{empty}\bigskip

%\noindent - {\bf Todas as resoluções devem incluir os cálculos e raciocínios usados para obter a solução.}

%\noindent - {\bf A prova pode ser feita a lápis, no entanto, os resultados e respostas finais devem ser escritos com caneta esferográfica de tinta azul ou preta}

\noindent - {\bf Não é permitido o uso de calculadora de qualquer natureza ou quaisquer outros aparelhos eletrônicos, sob pena de atribuir-se nota {\bf zero} à avaliação.}

\noindent - {\bf Não é permitido o uso de qualquer material alheio à avaliação.}

\noindent - {\bf Cada questão vale $1$ ponto, totalizando $10$ pontos.}
%\noindent - Há um ponto extra distribuido na avaliação.

%\noindent - Não é necessário grampear as avaliações.

%\noindent - Escreva as respostas em apenas um lado de cada folha.
%\noindent -  {\bf Mantenha as respostas em sequência.}

\noindent -  {\bf No cabeçalho, \underline{escreva seu nome completo}.}

%\noindent -  {\bf No rodapé direito, confira a ordem das folhas, e escreva \underline{``Página $n$ de $m$''}, para $n$ e $m \in \N$.}

%\noindent -  {\bf Se for o caso, consideraremos valores já calculados em outras questões, basta indicar e justificar.}
%Admita que $i^2 = -1$.
\begin{enumerate}

%qinicio
\item Determine o valor de $k$, de modo que $z=\left(\dfrac{k}{2}-\dfrac{1}{2} \right) + i$ seja um número imaginário puro.

\begin{enumerate}
\item $-\dfrac{1}{2}$
\item $-1$
\item $0$
\item $\dfrac{1}{2}$
\item $1$
\end{enumerate}
%qfim

%qinicio
\item A expressão $i^{13} + i^{15}$ é igual a:

\begin{enumerate}
	\item $0$ %gabarito
	\item $i$
	\item $-i$
	\item $-2i$
	\item $3i$
\end{enumerate}
%qfim
%qinicio
\item O módulo de $z = \dfrac{1}{i^{36}}$ é:

\begin{enumerate}
	\item $3$
	\item $1$ %gabarito
	\item $2$
	\item $\dfrac{1}{36}$
	\item $36$
\end{enumerate}
%qfim
%qinicio
\item Se $(1+ai)(b-i) = 5+5i$, com $a,b \in \mathbb{R}$, então $a$ e $b$ são raízes da equação

\begin{enumerate}
	\item $x^2-x-6=0$
	\item $x^2-5x-6=0$
	\item $x^2+x-6=0$
	\item $x^2+5x+6=0$
	\item $x^2-5x+6=0$ %gabarito
\end{enumerate}
%qfim
%qinicio
\item Sejam $z_1$ e $z_2$ as soluções da equação $x^2+x+1=0$. Determine $z_1^2+z_2^2$:

\begin{enumerate}
	\item $1$
	\item $-1$ %gabarito
	\item $2$
	\item $i$
	\item $-i$
\end{enumerate}
%qfim
%qinicio
\item A figura mostra, no plano complexo, o círculo de centro na origem e raio $1$ e as imagens de cinco números complexos. O complexo $\dfrac{1}{z}$ é igual a:
	\begin{center}
	\begin{tikzpicture}
		%\draw[help lines, color=gray!30, dashed] (-1.9,-1.9) grid (1.9,1.9);
		\draw[->, thick] (-2,0)--(2,0) node[right]{$Re$};
		\draw[->, thick] (0,-2)--(0,2) node[above]{$Im$};
		\draw (0,0) circle (30pt);
		\filldraw[black] (0.5,0.5) circle (1pt) node[below]{$z$};
		\filldraw[black] (1.5,1.5) circle (1pt) node[right]{$w$};
		\filldraw[black] (-1.5,1.5) circle (1pt) node[right]{$r$};
		\filldraw[black] (-1.5,-1.5) circle (1pt) node[right]{$s$};
		\filldraw[black] (1.5,-1.5) circle (1pt) node[right]{$t$};
	\end{tikzpicture}
	\end{center}

\begin{enumerate}
	\item $z$
	\item $w$
	\item $r$
	\item $s$
	\item $t$ %gabarito
\end{enumerate}
%qfim
%qinicio
\item Escrevendo o número complexo $z=\dfrac{1}{1-i} + \dfrac{1}{1+i}$ na forma algébrica, tem-se:

\begin{enumerate}
	\item $1-i$
	\item $i-1$
	\item $1+i$
	\item $i$
	\item $1$ %gabarito
\end{enumerate}
%qfim
%qinicio
\item Seja $z$ um número complexo de módulo $2$ e argumento $120°$. Nessas circunstâncias, o valor de $\overline{z}$ é:

\begin{enumerate}
	\item $2-2i\sqrt{3}$
	\item $2+2i\sqrt{3}$
	\item $-1-i\sqrt{3}$ %gabarito
	\item $-1+i\sqrt{3}$
	\item $1+i\sqrt{3}$
\end{enumerate}
%qfim
%qinicio
\item Seja $z=2+2i$. A forma trigonométrica de $z$ é:

\begin{enumerate}
	\item $\cis{2\sqrt{2}}{45°} $ %gabarito
	\item $\cis{2\sqrt{2}}{315°} $
	\item $\cis{4}{45°} $
	\item $\cis{\sqrt{2}}{135°} $
	\item $\cis{\sqrt{2}}{315°} $
\end{enumerate}
%qfim
%qinicio
\item O polígono $ABCDE$ da figura é um pentágono regular inscrito no círculo unitário de centro na origem do plano de Argand-Gauss.
	\begin{center}
	\usetikzlibrary{shapes.geometric}
	\begin{tikzpicture}
		%\draw[help lines, color=gray!30, dashed] (-1.9,-1.9) grid (1.9,1.9);
		\draw[->, thick] (-2,0)--(2,0) node[right]{$Re$};
		\draw[->, thick] (0,-2)--(0,2) node[above]{$Im$};
		\draw (0,0) circle (40pt);
		\node[regular polygon, regular polygon sides=5, draw, inner sep=23pt] (p) at (0,0) {};
		\node[draw,shape=circle,fill=black, inner sep=2pt] (p1) at (p.corner 1);
		\node[above right] at (p1) {$A$};
		\node[draw,shape=circle,fill=black, inner sep=2pt] (p2) at (p.corner 2);
		\node[left] at (p2) {$B$};
		\node[draw,shape=circle,fill=black, inner sep=2pt] (p3) at (p.corner 3);
		\node[below left] at (p3) {$C$};
		\node[draw,shape=circle,fill=black, inner sep=2pt] (p4) at (p.corner 4);
		\node[below right] at (p4) {$D$};
		\node[draw,shape=circle,fill=black, inner sep=2pt] (p5) at (p.corner 5);
		\node[right] at (p5) {$E$};
	\end{tikzpicture}
	\end{center}
Nessas condições, o argumento do complexo de afixo $A$ é igual a:

\begin{enumerate}
	\item $36°$
	\item $30°$
	\item $28°$
	\item $22°$
	\item $18°$ %gabarito
\end{enumerate}
%qfim

\iffalse{
\item Classifique os números a seguir em imaginário ou real. Se for o caso, indique quando o número for imaginário puro.
	\begin{enumerate}
		\item $\pi i$
		\item $i^{271828}-\overline{i}$
	\end{enumerate}

\item Sabendo que $z = \dfrac{\sqrt{3}}{2} + \dfrac{i}{2}$, marque, no plano de Argand-Gauss, os afixos dos números:
	\begin{enumerate}
		\item $z^3$
		\item $z^{127}$
	\end{enumerate}
	\begin{center}
	\begin{tikzpicture}
		\draw[help lines, color=gray!30, dashed] (-1.9,-1.9) grid (1.9,1.9);
		\draw[->, thick] (-2,0)--(2,0) node[right]{$Re$};
		\draw[->, thick] (0,-2)--(0,2) node[above]{$Im$};
	\end{tikzpicture}
	\end{center}
	
\item Determine para quais valores de $x$ a expressão $\dfrac{3+i}{x-i}$ a seguir é um número real.
	
\item Determine as raízes sextas de $\sqrt{3} - i$.

\item Dois números complexos $z$ e $w$ têm módulos iguais a $1$ e argumentos iguais a $35$° e $40$°, respectivamente. Determine os menores valores naturais de $m$ e $n$ para os quais se tem $z^m = w^n$.
\end{enumerate}
\clearpage
\mbox{~}
\clearpage
\mbox{~} }\fi

\end{enumerate}
\end{document}
